\chapter{Cecilia Payne, PhD.}

\lettrine[lines=2]{C}{ecilia Payne se} sentía feliz: ya era PhD en
Astronomía. A pesar de las interrupciones de Antonia y de los
\emph{bugs} del \annielogo{}\footnote{Para mayor referencia, véase el
  \href{https://drive.google.com/file/d/1QVz8UakYxqXMUqMXYt0nZgtviCBPOn8g/view?usp=sharing}{manual
    del programa \annielogo{}} (\emph{<<ANálisis Numérico de la
    Información Espectral>>}).} fue capaz de conquistar, en una
reflexiva epifanía, la significación profunda de las diferencias
espectrales entre las estrellas, gracias a lo cual pudo calcular las
proporciones relativas de sus elementos químicos constituyentes. El
título de su trabajo era suficientemente modesto: \emph{<<Atmósferas
  estelares: una contribución al estudio observacional de la alta
  temperatura en las capas inversoras de las
  estrellas>>}\footnote{Payne,
  C. H. (1925). \emph{\href{http://adsabs.harvard.edu/full/1925PhDT.........1P}{Stellar
      Atmospheres; a Contribution to the Observational Study of High
      Temperature in the Reversing Layers of Stars}} (PhD
  Thesis). Radcliffe College.}, pero ya sentía que, en no muchos años,
sería saludado como el trabajo singular más importante de la historia
de la Astronomía.

Cecilia respiró hondamente y, tras un largo suspiro, recordó sus
conversaciones con \director{} ---director del Observatorio de
Harvard--- durante las últimas etapas de la elaboración de su
monografía. <<¡Conversaciones..!>> ---pen\-só--- <<Más bien
monólogos...>>. Monólogos en los cuales \director{} la convenció
finalmente de mitigar sus resultados, ciertamente revolucionarios, con
los cuales demostraba que las estrellas estaban compuestas mayormente
por Hidrógeno. A \director{}, como al resto de los sabios de esos
años, le parecía que la composición química del Sol \emph{debía} ser
igual a la terrestre, y por eso Cecilia \emph{debía} morigerar sus
resultados... Trató de recordar en qué momento preciso la había
convencido; por supuesto, no lo logró, y nunca lo lograría. Así eran
los monólogos de \director{}: largas retahílas sintácticas de palabras
con una apariencia superficial de lógica, que luego en el recuerdo
---incluso inmediato--- se difuminaban en una nube de humo. <<¿Cómo
fue que le dije que sí?>> ---se preguntó Cecilia, acaso más
retóricamente que buscando una precisión que, sabía, no podría lograr
en un asunto en el que mediara \director{}. Pero ya no importaba: en
breve su tesis en todo su contenido original resurgiría y sus fieles
resultados serían aplaudidos por generaciones de astrónomos,
profesionales y aficionados, sin posibilidad alguna de quedar ya
sepultados bajo una insoportable montaña de humo denso.

Cecilia volvió a suspirar hondamente. Estaba sentada en un amplio
banco de los jardines de Harvard. El otoño apenas comenzaba y la suave
luz de un fresco mediodía se colaba entre los huecos de las hojas del
añoso y vital árbol a cuya sombra el banco estaba. Elevó su rostro y
recibió en sus párpados cerrados la delicada luz solar filtrada entre
las hojas, que sentía apenas como un brillo cálido acariciándola. Y
mientras pensaba que por fin sabía de qué estaba hecho el Sol, se
sorprendió una vez más al descubrir la dicha que ese conocimiento le
deparaba. Trató de distinguir, en ella, alguna nota de vanidad u
orgullo; pero no: con perplejidad y gratitud comprobó que el hecho
mismo de conocer, de saber, la hacía feliz. Siempre había sido una
curiosa, una inquieta: ahora podía comprobar que, sencillamente, lo
seguía siendo pero con el nombre de astrónoma profesional. Seguía
siendo aquella misma chiquilla que sentía una irresistible atracción
por los misterios del mundo y de la fantasía: la clara luz del Sol
vibrando en el agua durante el día y los oscuros espectros de Poe en
la profunda noche de su cuarto infantil; los íntimos secretos de la
luz de las estrellas palpitando para ella en los espectros de
absorción en su adulta oficina.

Se atrevió a abrir los ojos. Los huecos entre las hojas eran tan
pequeños que ningún rayo de Sol alcanzaba para cegarla. Antes bien,
parecían puntitos de luz en un fondo vegetal: estrellas titilando en
una noche verde oscuro. Bajó sus ojos al suelo, y vio la luz del Sol
en multitud de manchas moviéndose en el piso, en el banco, en sus
manos. Cecilia miraba ensoñadoramente el movimiento armónico y
delicado de esas manchas y sintió que, si bien era feliz, también una
sombra comenzaba a pesar en su alma: ahora que había terminado la
tesis que había colmado su tiempo y su energía los últimos años, ¿qué?
¿Qué seguía ahora? Una sensación de vacío había comenzado, lentamente,
a inquietarla durante los días anteriores.

Las manchas a sus pies, indiferentes a sus cavilaciones, seguían su
movimiento caprichoso. Cecilia casi ya no reparaba en ellas, hasta que
un conjunto de las mismas llamó imperceptiblemente su
atención. Parecían moverse juntas. No sólo eso: parecían formar una
suerte de patrón. Cecilia se frotó los ojos: parecían dispuestas según
un arreglo matricial. Su sorpresa se convirtió en alarma al comprobar
que esas manchas formaban, con una claridad mágica, un símbolo
numérico.

\begin{figure}[ht]
  \centering
  \includegraphics[width=.9\textwidth]{imagenes/digitos}
  \caption{Las misteriosas manchas de luz solar que alarmaron a
    Cecilia en el suelo de los jardines de Harvard un mediodía
    otoñal.}
  \label{fig:manchitas}
\end{figure}

Se irguió en el banco; sus ojos se clavaron, tan abiertos e inquietos
como su curiosidad, en esas manchas de luz ordenadas y
demenciales. Cecilia era tan joven que ni se le ocurrió la posibilidad
de haberse vuelto loca: su cabeza se hizo un súbito remolino en busca
de una explicación racional. Instintivamente echó su cabeza hacia
atrás, elevando sus ojos al cielo, y pudo apreciar con sorpresa que,
de una ventana en el primer piso del edificio que se encontraba a su
espalda se asomaba temerariamente un cuerpo que sostenía en sus manos
un pequeño y alargado objeto, del cual parecían salir rayos de
luz. Entrecerró los ojos tratando de reconocer a la persona asomada
cuando recordó súbitamente a quien pertenecía la oficina cuya ventana
enmarcaba ese cuerpo.

---¡Cecilia! ---la voz de Antonia Maury resonó, clara y socarrona, en
los jardines de Harvard---. ¿Qué hacés ahí sentada, con cara de
pavota?  Vení, subí que te muestro lo que acaba de inventar mi
inteligencia superior.

Cecilia sintió que una sonrisa casi le rompía la cara de lado a
lado. Rápida y feliz se levantó, y decididamente se dirigió a la
puerta que la conduciría, a través de los interminables pasillos de
Harvard, a la oficina de Antonia y a una nueva aventura.



%%% Local Variables:
%%% mode: latex
%%% TeX-master: "../libro"
%%% End:
